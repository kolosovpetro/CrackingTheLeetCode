\documentclass[12pt,letterpaper,oneside,reqno]{amsart}
\usepackage{amsfonts}
\usepackage{amsmath}
\usepackage{amssymb}
\usepackage{amsthm}
\usepackage{float}
\usepackage[font=small,labelfont=bf]{caption}
\usepackage[left=1in,right=1in,bottom=1in,top=1in]{geometry}
\usepackage[pdfpagelabels,hyperindex,colorlinks=true,linkcolor=blue,urlcolor=magenta,citecolor=green]{hyperref}

%--------Meta Data: Fill in your info------
\title[Cracking the LeetCode]{Cracking the LeetCode}
\author[Petro Kolosov]{Petro Kolosov}
\keywords{Data structures, Algorithms, Computer science, Coding Interview, LeetCode solutions}
\date{\today}
\hypersetup{
    pdftitle={Cracking the LeetCode},
    pdfsubject={Data structures, Algorithms, Computer science, Coding Interview, LeetCode solutions},
    pdfauthor={Petro Kolosov},
    pdfkeywords={Data structures, Algorithms, Computer science, Coding Interview, LeetCode solutions}
}
\begin{document}
    \begin{abstract}
        In the document the solutions to LeetCode problems are collected and explained in the details.
        Generally, this book aims an auditory such that preparing to the coding interviews.
    \end{abstract}
    \maketitle
    \tableofcontents


    \section{Arrays}\label{sec:arrays}


    \section{Dynamic Programming}\label{sec:dynamic-programming}


    \section{Fast and Slow Pointers}\label{sec:fast-and-slow-pointers}


    \section{Binary Search}\label{sec:binary-search}


    \section{BFS}\label{sec:bfs}


    \section{DFS}\label{sec:dfs}

    \bibliographystyle{unsrt}
    \bibliography{CreackingTheLeetCode_refs}
\end{document}